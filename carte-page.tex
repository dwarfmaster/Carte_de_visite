
%% $Id: carte-page-img-fgp.tex,v 1.6 2003/11/03 00:39:28 gilles Exp gilles $
\documentclass[english]{article}
\usepackage[T1]{fontenc}
\usepackage[latin1]{inputenc}
\usepackage{geometry}
\geometry{verbose,a4paper,tmargin=10mm,bmargin=10mm,lmargin=15mm,rmargin=15mm,headheight=0cm,headsep=0cm,footskip=0cm}
\pagestyle{empty}
\setlength\parskip{\medskipamount}
\setlength\parindent{0pt}
\usepackage{color}
\usepackage{graphicx}

\usepackage{multicol}
\makeatother

\providecommand{\LyX}{L\kern-.1667em\lower.25em\hbox{Y}\kern-.125emX\@}
\providecommand{\tabularnewline}{\\}

\usepackage{letterspace}
\providecommand{\xw}[1]{\letterspace to 80mm{#1}}

\usepackage{eso-pic}
\newcommand\BackgroundPic{
	\put(0,0){
		\parbox[b][\paperheight]{\paperwidth}{%
			\vfill
			\centering
			\includegraphics[width=\paperwidth,height=\paperheight]{images/bg.ps}%
			\vfill
}}}


\makeatother
\begin{document}
\setlength{\multicolsep}{1cm}
\setlength{\columnseprule}{0mm}
\setcounter{unbalance}{1}
\setcounter{columnbadness}{0}
\begin{multicols}{2}
[\section*{Cartes de visite de Luc Chabassier}]
\raggedcolumns
\noindent \input{carte-fgp.tex}
\noindent \input{carte-fgp.tex}
\noindent \input{carte-fgp.tex}
\noindent \input{carte-fgp.tex}
\noindent \input{carte-fgp.tex}
\noindent \input{carte-fgp.tex}
\noindent \input{carte-fgp.tex}
\noindent \input{carte-fgp.tex}

\end{multicols}
\end{document}

